% Options for packages loaded elsewhere
\PassOptionsToPackage{unicode}{hyperref}
\PassOptionsToPackage{hyphens}{url}
%
\documentclass[
  ignorenonframetext,
]{beamer}
\usepackage{pgfpages}
\setbeamertemplate{caption}[numbered]
\setbeamertemplate{caption label separator}{: }
\setbeamercolor{caption name}{fg=normal text.fg}
\beamertemplatenavigationsymbolsempty
% Prevent slide breaks in the middle of a paragraph
\widowpenalties 1 10000
\raggedbottom
\setbeamertemplate{part page}{
  \centering
  \begin{beamercolorbox}[sep=16pt,center]{part title}
    \usebeamerfont{part title}\insertpart\par
  \end{beamercolorbox}
}
\setbeamertemplate{section page}{
  \centering
  \begin{beamercolorbox}[sep=12pt,center]{part title}
    \usebeamerfont{section title}\insertsection\par
  \end{beamercolorbox}
}
\setbeamertemplate{subsection page}{
  \centering
  \begin{beamercolorbox}[sep=8pt,center]{part title}
    \usebeamerfont{subsection title}\insertsubsection\par
  \end{beamercolorbox}
}
\AtBeginPart{
  \frame{\partpage}
}
\AtBeginSection{
  \ifbibliography
  \else
    \frame{\sectionpage}
  \fi
}
\AtBeginSubsection{
  \frame{\subsectionpage}
}
\usepackage{amsmath,amssymb}
\usepackage{iftex}
\ifPDFTeX
  \usepackage[T1]{fontenc}
  \usepackage[utf8]{inputenc}
  \usepackage{textcomp} % provide euro and other symbols
\else % if luatex or xetex
  \usepackage{unicode-math} % this also loads fontspec
  \defaultfontfeatures{Scale=MatchLowercase}
  \defaultfontfeatures[\rmfamily]{Ligatures=TeX,Scale=1}
\fi
\usepackage{lmodern}
\ifPDFTeX\else
  % xetex/luatex font selection
\fi
% Use upquote if available, for straight quotes in verbatim environments
\IfFileExists{upquote.sty}{\usepackage{upquote}}{}
\IfFileExists{microtype.sty}{% use microtype if available
  \usepackage[]{microtype}
  \UseMicrotypeSet[protrusion]{basicmath} % disable protrusion for tt fonts
}{}
\makeatletter
\@ifundefined{KOMAClassName}{% if non-KOMA class
  \IfFileExists{parskip.sty}{%
    \usepackage{parskip}
  }{% else
    \setlength{\parindent}{0pt}
    \setlength{\parskip}{6pt plus 2pt minus 1pt}}
}{% if KOMA class
  \KOMAoptions{parskip=half}}
\makeatother
\usepackage{xcolor}
\newif\ifbibliography
\usepackage{color}
\usepackage{fancyvrb}
\newcommand{\VerbBar}{|}
\newcommand{\VERB}{\Verb[commandchars=\\\{\}]}
\DefineVerbatimEnvironment{Highlighting}{Verbatim}{commandchars=\\\{\}}
% Add ',fontsize=\small' for more characters per line
\usepackage{framed}
\definecolor{shadecolor}{RGB}{248,248,248}
\newenvironment{Shaded}{\begin{snugshade}}{\end{snugshade}}
\newcommand{\AlertTok}[1]{\textcolor[rgb]{0.94,0.16,0.16}{#1}}
\newcommand{\AnnotationTok}[1]{\textcolor[rgb]{0.56,0.35,0.01}{\textbf{\textit{#1}}}}
\newcommand{\AttributeTok}[1]{\textcolor[rgb]{0.13,0.29,0.53}{#1}}
\newcommand{\BaseNTok}[1]{\textcolor[rgb]{0.00,0.00,0.81}{#1}}
\newcommand{\BuiltInTok}[1]{#1}
\newcommand{\CharTok}[1]{\textcolor[rgb]{0.31,0.60,0.02}{#1}}
\newcommand{\CommentTok}[1]{\textcolor[rgb]{0.56,0.35,0.01}{\textit{#1}}}
\newcommand{\CommentVarTok}[1]{\textcolor[rgb]{0.56,0.35,0.01}{\textbf{\textit{#1}}}}
\newcommand{\ConstantTok}[1]{\textcolor[rgb]{0.56,0.35,0.01}{#1}}
\newcommand{\ControlFlowTok}[1]{\textcolor[rgb]{0.13,0.29,0.53}{\textbf{#1}}}
\newcommand{\DataTypeTok}[1]{\textcolor[rgb]{0.13,0.29,0.53}{#1}}
\newcommand{\DecValTok}[1]{\textcolor[rgb]{0.00,0.00,0.81}{#1}}
\newcommand{\DocumentationTok}[1]{\textcolor[rgb]{0.56,0.35,0.01}{\textbf{\textit{#1}}}}
\newcommand{\ErrorTok}[1]{\textcolor[rgb]{0.64,0.00,0.00}{\textbf{#1}}}
\newcommand{\ExtensionTok}[1]{#1}
\newcommand{\FloatTok}[1]{\textcolor[rgb]{0.00,0.00,0.81}{#1}}
\newcommand{\FunctionTok}[1]{\textcolor[rgb]{0.13,0.29,0.53}{\textbf{#1}}}
\newcommand{\ImportTok}[1]{#1}
\newcommand{\InformationTok}[1]{\textcolor[rgb]{0.56,0.35,0.01}{\textbf{\textit{#1}}}}
\newcommand{\KeywordTok}[1]{\textcolor[rgb]{0.13,0.29,0.53}{\textbf{#1}}}
\newcommand{\NormalTok}[1]{#1}
\newcommand{\OperatorTok}[1]{\textcolor[rgb]{0.81,0.36,0.00}{\textbf{#1}}}
\newcommand{\OtherTok}[1]{\textcolor[rgb]{0.56,0.35,0.01}{#1}}
\newcommand{\PreprocessorTok}[1]{\textcolor[rgb]{0.56,0.35,0.01}{\textit{#1}}}
\newcommand{\RegionMarkerTok}[1]{#1}
\newcommand{\SpecialCharTok}[1]{\textcolor[rgb]{0.81,0.36,0.00}{\textbf{#1}}}
\newcommand{\SpecialStringTok}[1]{\textcolor[rgb]{0.31,0.60,0.02}{#1}}
\newcommand{\StringTok}[1]{\textcolor[rgb]{0.31,0.60,0.02}{#1}}
\newcommand{\VariableTok}[1]{\textcolor[rgb]{0.00,0.00,0.00}{#1}}
\newcommand{\VerbatimStringTok}[1]{\textcolor[rgb]{0.31,0.60,0.02}{#1}}
\newcommand{\WarningTok}[1]{\textcolor[rgb]{0.56,0.35,0.01}{\textbf{\textit{#1}}}}
\usepackage{graphicx}
\makeatletter
\def\maxwidth{\ifdim\Gin@nat@width>\linewidth\linewidth\else\Gin@nat@width\fi}
\def\maxheight{\ifdim\Gin@nat@height>\textheight\textheight\else\Gin@nat@height\fi}
\makeatother
% Scale images if necessary, so that they will not overflow the page
% margins by default, and it is still possible to overwrite the defaults
% using explicit options in \includegraphics[width, height, ...]{}
\setkeys{Gin}{width=\maxwidth,height=\maxheight,keepaspectratio}
% Set default figure placement to htbp
\makeatletter
\def\fps@figure{htbp}
\makeatother
\setlength{\emergencystretch}{3em} % prevent overfull lines
\providecommand{\tightlist}{%
  \setlength{\itemsep}{0pt}\setlength{\parskip}{0pt}}
\setcounter{secnumdepth}{-\maxdimen} % remove section numbering
\ifLuaTeX
  \usepackage{selnolig}  % disable illegal ligatures
\fi
\IfFileExists{bookmark.sty}{\usepackage{bookmark}}{\usepackage{hyperref}}
\IfFileExists{xurl.sty}{\usepackage{xurl}}{} % add URL line breaks if available
\urlstyle{same}
\hypersetup{
  pdftitle={R: Fundamental Concepts with RStudio},
  hidelinks,
  pdfcreator={LaTeX via pandoc}}

\title{R: Fundamental Concepts with RStudio}
\author{}
\date{\vspace{-2.5em}}

\begin{document}
\frame{\titlepage}

\begin{frame}
Last Updated: 2025-10-09
\end{frame}

\begin{frame}{RStudio}
\protect\hypertarget{rstudio}{}
R is great for interactive computational analysis, allowing for easy
exploration and iteration. Using an IDE - Integrated Development
Environment - such as RStudio, facilitates this interactivity, providing
you with easy access to a place to quickly code (the console), a place
to write saveable scripts (source editor), a place to view the objects
that you create (environment pane), and a place to interact with your
file system, visual outputs from your code, and help documentation (the
miscellaneous pane, for lack of a better word).

\includegraphics{https://raw.githubusercontent.com/csc-ubc-okanagan/workshops/main/docs/assets/images/rstudio.png}
\end{frame}

\begin{frame}[fragile]{Simple Math}
\protect\hypertarget{simple-math}{}
At it's most basic, R is a glorified calculator, capable of any
operation might wish to throw at it\ldots{}

\begin{Shaded}
\begin{Highlighting}[]
\CommentTok{\# addition}
\DecValTok{2} \SpecialCharTok{+} \DecValTok{2}
\end{Highlighting}
\end{Shaded}

\begin{verbatim}
## [1] 4
\end{verbatim}

\begin{Shaded}
\begin{Highlighting}[]
\CommentTok{\# subtraction}
\DecValTok{3} \SpecialCharTok{{-}} \DecValTok{2}
\end{Highlighting}
\end{Shaded}

\begin{verbatim}
## [1] 1
\end{verbatim}

\begin{Shaded}
\begin{Highlighting}[]
\CommentTok{\# mulitplication}
\DecValTok{3} \SpecialCharTok{*} \DecValTok{3}
\end{Highlighting}
\end{Shaded}

\begin{verbatim}
## [1] 9
\end{verbatim}

\begin{Shaded}
\begin{Highlighting}[]
\CommentTok{\# division}
\DecValTok{4} \SpecialCharTok{/} \DecValTok{2}
\end{Highlighting}
\end{Shaded}

\begin{verbatim}
## [1] 2
\end{verbatim}

\begin{Shaded}
\begin{Highlighting}[]
\CommentTok{\# square root}
\FunctionTok{sqrt}\NormalTok{(}\DecValTok{9}\NormalTok{)}
\end{Highlighting}
\end{Shaded}

\begin{verbatim}
## [1] 3
\end{verbatim}

\begin{Shaded}
\begin{Highlighting}[]
\CommentTok{\# log 10}
\FunctionTok{log10}\NormalTok{(}\DecValTok{100}\NormalTok{)}
\end{Highlighting}
\end{Shaded}

\begin{verbatim}
## [1] 2
\end{verbatim}
\end{frame}

\begin{frame}[fragile]{Functions}
\protect\hypertarget{functions}{}
At least initially, everything you do in R will be applying functions to
data. In the above section, the mathematical operators, \texttt{sqrt()},
and \texttt{log10()}, are all functions. Functions take in data (or
values) and process them, providing an output. The output is displayed
in your console by default.

\begin{Shaded}
\begin{Highlighting}[]
\CommentTok{\# c() is a common function for concatenating things together}
\FunctionTok{c}\NormalTok{(}\DecValTok{1}\SpecialCharTok{:}\DecValTok{10}\NormalTok{)}
\end{Highlighting}
\end{Shaded}

\begin{verbatim}
##  [1]  1  2  3  4  5  6  7  8  9 10
\end{verbatim}
\end{frame}

\begin{frame}[fragile]{Variables}
\protect\hypertarget{variables}{}
While having outputs directed to the console is convenient for
interactive analysis, often we need to store data and / or outputs for
later use. We do this with variable assignmet, where the greater than
and dash are used to assign the values (object) on the right to the name
(variable) on the left.

\begin{Shaded}
\begin{Highlighting}[]
\NormalTok{my\_variable }\OtherTok{\textless{}{-}} \FunctionTok{c}\NormalTok{(}\DecValTok{1}\SpecialCharTok{:}\DecValTok{10}\NormalTok{)}
\end{Highlighting}
\end{Shaded}

You can then recall the values (object) associated with your
variable\ldots{}

\begin{Shaded}
\begin{Highlighting}[]
\NormalTok{my\_variable}
\end{Highlighting}
\end{Shaded}

\begin{verbatim}
##  [1]  1  2  3  4  5  6  7  8  9 10
\end{verbatim}

And plug it into functions, ie, do computations on it\ldots{}

\begin{Shaded}
\begin{Highlighting}[]
\NormalTok{my\_variable }\SpecialCharTok{*} \DecValTok{2}
\end{Highlighting}
\end{Shaded}

\begin{verbatim}
##  [1]  2  4  6  8 10 12 14 16 18 20
\end{verbatim}

\begin{quote}
When naming variables in R, keep in mind that variable names:

\begin{itemize}
\tightlist
\item
  Should first and foremost be meaningful. This is not a rule, just best
  practice.
\item
  Cannot start with a number or a dot followed by a number.
\item
  Cannot contain spaces or hyphens.
\item
  Can contain letters, numbers, dots, and underscores.
\end{itemize}

Additionally, some words are reserved and cannot be used, such as if and
for. More details can be found with \texttt{?make.names}
\end{quote}
\end{frame}

\begin{frame}[fragile]{Data Types \& Structures}
\protect\hypertarget{data-types-structures}{}
\begin{block}{Data Types}
\protect\hypertarget{data-types}{}
Data types are elemental data constructs that R is able to distinguish
between. The most common data types in R are \texttt{numeric},
\texttt{character}, and \texttt{boolean}. These come with intrinsic
properties, for example, numeric data can be computed (added, divided,
etc), character data can be strung together (character data, especially
groups of characters are referred to as strings), and boolean data
facilitates working with dichotomous values.

\begin{itemize}
\tightlist
\item
  \texttt{boolean} data are generally referred to as \texttt{logical}.
\item
  \texttt{numeric} data are subdivided into \texttt{double} and
  \texttt{integer}; the difference is generally of little significance,
  as R manages the distinctions for you, but be wary if you're dealing
  with particularly large numbers.
\item
  \texttt{character} data always needs to be wrapped in single or double
  quotations.
\end{itemize}

The function \texttt{typeof()}, will tell you what data type you
have\ldots{}

\begin{Shaded}
\begin{Highlighting}[]
\FunctionTok{typeof}\NormalTok{(}\DecValTok{2}\NormalTok{)}
\end{Highlighting}
\end{Shaded}

\begin{verbatim}
## [1] "double"
\end{verbatim}

\begin{Shaded}
\begin{Highlighting}[]
\FunctionTok{typeof}\NormalTok{(}\StringTok{"a"}\NormalTok{)}
\end{Highlighting}
\end{Shaded}

\begin{verbatim}
## [1] "character"
\end{verbatim}

\begin{Shaded}
\begin{Highlighting}[]
\FunctionTok{typeof}\NormalTok{(}\ConstantTok{TRUE}\NormalTok{)}
\end{Highlighting}
\end{Shaded}

\begin{verbatim}
## [1] "logical"
\end{verbatim}
\end{block}

\begin{block}{Data Structures}
\protect\hypertarget{data-structures}{}
Data structures are ways to hold multiple pieces of data together in a
meaningful way. R has four convenient ways of storing multiple pieces of
data: vectors, matrices, data frames, and lists.

\begin{block}{Vectors}
\protect\hypertarget{vectors}{}
The most basic is a \texttt{vector}. In fact, everything in R is a
vector, and all other data structures are composed of vectors in various
ways. It's convenient to think of a vector as a simple list of like
things; everything in a vector needs to be of the same data type. So, a
vector can be:

\begin{itemize}
\tightlist
\item
  numeric, holding numeric data;
\item
  character, holding character data; or
\item
  logical, holding boolean data.
\end{itemize}

A vector can be created with \texttt{c()}

\begin{Shaded}
\begin{Highlighting}[]
\CommentTok{\# numeric vector}
\FunctionTok{c}\NormalTok{(}\DecValTok{1}\SpecialCharTok{:}\DecValTok{10}\NormalTok{)}
\end{Highlighting}
\end{Shaded}

\begin{verbatim}
##  [1]  1  2  3  4  5  6  7  8  9 10
\end{verbatim}

\begin{Shaded}
\begin{Highlighting}[]
\CommentTok{\# character vector}
\FunctionTok{c}\NormalTok{(}\StringTok{"a"}\NormalTok{, }\StringTok{"b"}\NormalTok{, }\StringTok{"c"}\NormalTok{)}
\end{Highlighting}
\end{Shaded}

\begin{verbatim}
## [1] "a" "b" "c"
\end{verbatim}

\begin{Shaded}
\begin{Highlighting}[]
\CommentTok{\# logical vector}
\FunctionTok{c}\NormalTok{(}\ConstantTok{TRUE}\NormalTok{, }\ConstantTok{TRUE}\NormalTok{, }\ConstantTok{FALSE}\NormalTok{)}
\end{Highlighting}
\end{Shaded}

\begin{verbatim}
## [1]  TRUE  TRUE FALSE
\end{verbatim}
\end{block}

\begin{block}{Matrices}
\protect\hypertarget{matrices}{}
A matrix is a vector with 2 dimensions; while a vector only has a
length, a matrix is a grid with both length and width, or alternatively,
rows and columns.

You can create a matrix with the \texttt{matrix()} function, providing
it first with a series of values followed by an argument for the number
of rows or columns you'd like.

\begin{Shaded}
\begin{Highlighting}[]
\FunctionTok{matrix}\NormalTok{(}\DecValTok{1}\SpecialCharTok{:}\DecValTok{10}\NormalTok{, }\AttributeTok{nrow =} \DecValTok{2}\NormalTok{)}
\end{Highlighting}
\end{Shaded}

\begin{verbatim}
##      [,1] [,2] [,3] [,4] [,5]
## [1,]    1    3    5    7    9
## [2,]    2    4    6    8   10
\end{verbatim}

\begin{Shaded}
\begin{Highlighting}[]
\FunctionTok{matrix}\NormalTok{(}\DecValTok{1}\SpecialCharTok{:}\DecValTok{10}\NormalTok{, }\AttributeTok{ncol =} \DecValTok{2}\NormalTok{)}
\end{Highlighting}
\end{Shaded}

\begin{verbatim}
##      [,1] [,2]
## [1,]    1    6
## [2,]    2    7
## [3,]    3    8
## [4,]    4    9
## [5,]    5   10
\end{verbatim}

A matrix must be perfectly rectangular, that is, each column must be of
equal length. R will recycle values to ensure this condition is met.

\begin{Shaded}
\begin{Highlighting}[]
\CommentTok{\# R will recycle the 1 to complete the rectangle}
\FunctionTok{matrix}\NormalTok{(}\DecValTok{1}\SpecialCharTok{:}\DecValTok{11}\NormalTok{, }\AttributeTok{ncol =} \DecValTok{2}\NormalTok{)}
\end{Highlighting}
\end{Shaded}

\begin{verbatim}
## Warning in matrix(1:11, ncol = 2): data length [11] is not a sub-multiple or
## multiple of the number of rows [6]
\end{verbatim}

\begin{verbatim}
##      [,1] [,2]
## [1,]    1    7
## [2,]    2    8
## [3,]    3    9
## [4,]    4   10
## [5,]    5   11
## [6,]    6    1
\end{verbatim}

A matrix must also be atomic - the data type must be the same
throughout.
\end{block}

\begin{block}{Data frames}
\protect\hypertarget{data-frames}{}
In general, these will be the most likely data structures that you'll
encounter and the most familiar, as they are designed for tabular data
and look like a spreadsheet.

Data frames are collections of vectors, where each vector must be of the
same length, but can be of a different data type. If you import data
from an Excel or csv file, these will be read in as a data frame. You
can also create them with the \texttt{data.frame()} function.

\begin{Shaded}
\begin{Highlighting}[]
\FunctionTok{data.frame}\NormalTok{(}\AttributeTok{var\_1 =}\NormalTok{ letters,}
           \AttributeTok{var\_2 =} \DecValTok{1}\SpecialCharTok{:}\DecValTok{26}\NormalTok{)}
\end{Highlighting}
\end{Shaded}

\begin{verbatim}
##    var_1 var_2
## 1      a     1
## 2      b     2
## 3      c     3
## 4      d     4
## 5      e     5
## 6      f     6
## 7      g     7
## 8      h     8
## 9      i     9
## 10     j    10
## 11     k    11
## 12     l    12
## 13     m    13
## 14     n    14
## 15     o    15
## 16     p    16
## 17     q    17
## 18     r    18
## 19     s    19
## 20     t    20
## 21     u    21
## 22     v    22
## 23     w    23
## 24     x    24
## 25     y    25
## 26     z    26
\end{verbatim}
\end{block}

\begin{block}{Lists}
\protect\hypertarget{lists}{}
Like data frames, lists are collections of vectors. However, unlike data
frames, the length of these vectors can change. So far, this is the only
non-rectangular data structure that we have to work with.

If you are importing tabular data, you are unlikely to create a list
yourself, however, when performing operations (using functions) on your
data, the output, or result, might be in the form of a list. For
demonstration puropses, a list can be created with the \texttt{list()}
function.

\begin{Shaded}
\begin{Highlighting}[]
\FunctionTok{list}\NormalTok{(}\AttributeTok{item\_1 =}\NormalTok{ letters,}
     \AttributeTok{item\_2 =} \DecValTok{1}\SpecialCharTok{:}\DecValTok{10}\NormalTok{,}
     \AttributeTok{item\_3 =} \FunctionTok{c}\NormalTok{(}\ConstantTok{TRUE}\NormalTok{, }\ConstantTok{FALSE}\NormalTok{))}
\end{Highlighting}
\end{Shaded}

\begin{verbatim}
## $item_1
##  [1] "a" "b" "c" "d" "e" "f" "g" "h" "i" "j" "k" "l" "m" "n" "o" "p" "q" "r" "s"
## [20] "t" "u" "v" "w" "x" "y" "z"
## 
## $item_2
##  [1]  1  2  3  4  5  6  7  8  9 10
## 
## $item_3
## [1]  TRUE FALSE
\end{verbatim}
\end{block}

\begin{block}{Identifying Data Structures}
\protect\hypertarget{identifying-data-structures}{}
The function \texttt{class()} will tell you what kind of structure your
data are held in.

\begin{quote}
\texttt{class()}, when called on a vector, will report on the data type,
which should be read as the type of atomic vector you have. An array is
a multi-dimensional vector, a matrix is the special case where there are
only two dimensions.
\end{quote}

\begin{Shaded}
\begin{Highlighting}[]
\FunctionTok{class}\NormalTok{(}\FunctionTok{c}\NormalTok{(}\DecValTok{1}\SpecialCharTok{:}\DecValTok{10}\NormalTok{))}
\end{Highlighting}
\end{Shaded}

\begin{verbatim}
## [1] "integer"
\end{verbatim}

\begin{Shaded}
\begin{Highlighting}[]
\FunctionTok{class}\NormalTok{(}\FunctionTok{matrix}\NormalTok{(}\DecValTok{1}\SpecialCharTok{:}\DecValTok{10}\NormalTok{, }\AttributeTok{ncol =} \DecValTok{2}\NormalTok{))}
\end{Highlighting}
\end{Shaded}

\begin{verbatim}
## [1] "matrix" "array"
\end{verbatim}

\begin{Shaded}
\begin{Highlighting}[]
\FunctionTok{class}\NormalTok{(}\FunctionTok{data.frame}\NormalTok{(}\AttributeTok{var\_1 =}\NormalTok{ letters,}
           \AttributeTok{var\_2 =} \DecValTok{1}\SpecialCharTok{:}\DecValTok{26}\NormalTok{))}
\end{Highlighting}
\end{Shaded}

\begin{verbatim}
## [1] "data.frame"
\end{verbatim}

\begin{Shaded}
\begin{Highlighting}[]
\FunctionTok{class}\NormalTok{(}\FunctionTok{list}\NormalTok{(}\AttributeTok{item\_1 =}\NormalTok{ letters,}
     \AttributeTok{item\_2 =} \DecValTok{1}\SpecialCharTok{:}\DecValTok{10}\NormalTok{,}
     \AttributeTok{item\_3 =} \FunctionTok{c}\NormalTok{(}\ConstantTok{TRUE}\NormalTok{, }\ConstantTok{FALSE}\NormalTok{)))}
\end{Highlighting}
\end{Shaded}

\begin{verbatim}
## [1] "list"
\end{verbatim}
\end{block}
\end{block}
\end{frame}

\begin{frame}[fragile]{Getting Help}
\protect\hypertarget{getting-help}{}
There are five primary ways to get help (other than asking a friend)
with R.

The first is with the built in documentation, which can be accessed by
preceding a function name with a \texttt{?}.

\begin{Shaded}
\begin{Highlighting}[]
\NormalTok{?}\FunctionTok{sqrt}\NormalTok{()}
\end{Highlighting}
\end{Shaded}

These pages can be both enlightening and frustrating, as they assume
some pre-existing familiarity with certain concepts as well as reading
this kind of documentation.

The second is from forums. These are useful when you have a specific
issue that you're trying to resolve, and can often provide a quick fix,
but the explanations also often require a degree of familiarity to make
sense.

The third is blog posts and course supports posted on line. These
generally provide more context than forum posts, and are useful when
tyring to learn a new approach or a new library.

The fourth is journal publications that accompany the release of
specific packages or libraries. These offer in depth descriptions of the
packages and their arguments, often with examples.

The fourth is more in depth texts. Many of these are available publicly
online, and include titles such as:

\begin{itemize}
\tightlist
\item
  aRrgh: a newcomer's (angry) guide to R, by Tim Smith and Kevin Ushey -
  \url{http://arrgh.tim-smith.us/}
\item
  YaRrr! The Pirate's Guide to R, by Nathaniel D. Phillips -
  \url{https://bookdown.org/ndphillips/YaRrr/}
\item
  R for Graduate Students, by Wendy Huynh. -
  \url{https://bookdown.org/yih_huynh/Guide-to-R-Book/}
\item
  Advanced R, by Hadley Wickham -
  \url{http://adv-r.had.co.nz/Introduction.html}
\end{itemize}

And through the library, the
\href{https://resources.library.ubc.ca/page.php?details=oreilly-for-higher-education\&id=2460}{O'Reilly
platform} hosts a plethora of titles related to R.
\end{frame}

\end{document}
